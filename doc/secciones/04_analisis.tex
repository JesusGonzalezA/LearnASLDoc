\chapter{Analysis}

\section{Objectives}
The main objectives of the app to be developped, thinking of them as high-level requisites, are the following ones: 

\begin{enumerate}[label=\textbf{Obj\_\arabic*:}, align=left, leftmargin=*]
    \item The system will manage and keep all the information related to users, such as their credentials, done tests and stats.
    \item The system will update the users' stats automatically after every test. 
    \item The system will allow the users to manage their account.
    \item The system will allow the users to read their stats and tests.
    \item The system will manage the tests, being capable of creating them every time a user request to start one.
    \item The system will store the videos of the users signing and will manage its visibility (to hide the videos to other users so that privacy is achieved).
\end{enumerate}

\section{Requirements}
\subsection{Functional requirements}
\begin{enumerate}[label=\textbf{RF\_\arabic*}, align=left, leftmargin=*]
    \item Users management:
        \begin{enumerate}[label=\textbf{\theenumi.\arabic*}, align=left, leftmargin=*]
            \item The system will allow a non registered user be signed up in the the system.
            \item The system will store all the information relative to the user in order to be able to identify them before entering the app.
            \item The system will allow a registered user and identified to delete their account and information.
        \end{enumerate}
    \item Tests management: 
        \begin{enumerate}[label=\textbf{\theenumi.\arabic*}, align=left, leftmargin=*]
            \item The system will be able to generate all different types of tests:
                \begin{itemize}[noitemsep]
                    \item \textit{Mimic tests}
                    \item \textit{QA tests}
                    \item \textit{Option tests}
                        \begin{itemize}[noitemsep]
                            \item \textit{Word to video test}
                            \item \textit{Video to word test}
                        \end{itemize}
                    \item \textit{Error test} variant from all the tests above
                \end{itemize}
            \item The system will be able to generate tests depending on the difficulty level. The difficulty will be \textit{EASY}, \textit{MEDIUM} and \textit{HARD}, depending on the selected dataset, \textit{WLASL300},  \textit{WLASL1000},  \textit{WLASL2000} respectively.
            \item The system will allow the user to review an already done test and its results.
            \item The system will protect the review of tests, so that only the user who did the test is able to review it.
            \item The system will allow the user to delete all the done tests.
        \end{enumerate}
    \item Analytics management: The system will create stats from the tests done by the user until the moment the stats are required.
        \begin{enumerate}[label=\textbf{\theenumi.\arabic*}, align=left, leftmargin=*]
            \item The system will create stats and display diagrams of the following topics:
                \begin{enumerate}
                    \item Success percent.
                    \item Percent of learnt words.
                    \item Days the user has started a test in the app.
                    \item New learnt words.
                \end{enumerate}
            \item The system will allow the user to filter the stats by difficulty and date.
        \end{enumerate}
\end{enumerate}

\subsection{Non-Functional requirements}
\begin{enumerate}[label=\textbf{RNF\_\arabic*}, align=left, leftmargin=*]
    \item \textbf{Open source:} The application should be open source, due to its social interest.
    \item \textbf{Accessibility: } The application should be accesible. Therefore, a well-tested UI library should be used and there will be no action that depends on sounds.
    \item \textbf{Usability:} The application should be responsive.
    \item \textbf{Maintenability:} The application core should have a clean architecture, so that the maintenance of the application is guaranteed to be a simpler.
    \item \textbf{Testability:} The main core of the application should be tested using unit tests.
    \item \textbf{Privacy:} The application will require authorization and authentication in order to let a user interact with the system. The user will only be granted to access their own resources.
\end{enumerate}

\section{User's stories}
In order to write the stories and follow their development, I created a project in GitHub. You can access it here \url{https://github.com/JesusGonzalezA/LearnASL/projects/3}. \\

I divided the user stories into three main categories, according to the Functional Requirements:
\begin{enumerate}[label=\textbf{US\_\arabic*}, align=left, leftmargin=*]
    \item \textbf{User management} 
        \begin{enumerate}[label=\textbf{\theenumi.\arabic*}, align=left, leftmargin=*]
            \item As a non signer, I want to register in the app to learn American Sign Language.
            \item As a user, I want to login in the app.
            \item As a user, I want to change my password.
            \item As a user, I want to delete my account.
            \item As a user who does not remember their password, I want to be able to recover my account.
            \item As an already registered user, I want to update my email.
        \end{enumerate}
    \item \textbf{Test management} 
        \begin{enumerate}[label=\textbf{\theenumi.\arabic*}, align=left, leftmargin=*]
            \item As a user, I want to start a new test.
            \item As a user, I want to select the difficulty of a test.
            \item As a user, I want to customize the number of questions of a test.
            \item As a user, I want to know my result in the test.
            \item As a user, I want to see in which question I am on.
            \item As a user, I want to review a test done in the past.
            \item As a user, I want to see a list with all the tests I've made.
        \end{enumerate}
    \item \textbf{Stats management}
        \begin{enumerate}[label=\textbf{\theenumi.\arabic*}, align=left, leftmargin=*]
            \item As a user, I want to select the interval of time of my stats.
            \item As a user, I want to review my stats.
            \item As a user, I want to delete my history of tests.
        \end{enumerate}
\end{enumerate}

\subsection{Milestones}
Each milestone represents a group of user stories that can be grouped in order to present a value increase.

\begin{enumerate}[label=\textbf{MIL\_\arabic*}, align=left, leftmargin=*]
    \item Project set up
        \begin{enumerate}[label=\textbf{\theenumi.\arabic*}, align=left, leftmargin=*]
            \item Both backend and frontend projects are initialized.
            \item The test and production environments are set up.
            \item The project is containerized.
            \item The Github actions are prepared.
        \end{enumerate}
    \item Authentication
        \begin{enumerate}[label=\textbf{\theenumi.\arabic*}, align=left, leftmargin=*]
            \item A user can register/login/log out/sign off/change password/change email. Refers to \textit{US\_1}.
            \item There are private/public routes and screens.
        \end{enumerate}
    \item The app is designed
        \begin{enumerate}[label=\textbf{\theenumi.\arabic*}, align=left, leftmargin=*]
            \item A good Readme is created.
            \item Figma \cite{Figma} Low-Fidelity design.
            \item Figma \cite{Figma} High-Fidelity design.
        \end{enumerate}
    \item A web version of the app is avalaible
        \begin{enumerate}[label=\textbf{\theenumi.\arabic*}, align=left, leftmargin=*]
            \item React version of the application with faked data.
        \end{enumerate}
    \item The app can create tests
        \begin{enumerate}[label=\textbf{\theenumi.\arabic*}, align=left, leftmargin=*]
            \item The app can create tests. Refers to \textit{US\_2.1}.
            \item The user can upload a video. Refers to \textit{US\_2}.
            \item The user can start a new test. Refers to \textit{US\_2.1}.
        \end{enumerate}
    \item The app saves info
        \begin{enumerate}[label=\textbf{\theenumi.\arabic*}, align=left, leftmargin=*]
            \item A user can manage information about themselves. Refers to \textit{US\_1, US\_2, US\_3}.
            \item A user can watch past tests. Refers to \textit{US\_2.4, US\_2.6}.
            \item A user can delete old tests. Refers to \textit{US\_3.3}.
        \end{enumerate}
    \item Graphical stats
        \begin{enumerate}[label=\textbf{\theenumi.\arabic*}, align=left, leftmargin=*]
            \item The user can see the data in a graphical way. Refers to \textit{US\_3.1, US\_3.2}.
        \end{enumerate}
    \item The app validates the videos
        \begin{enumerate}[label=\textbf{\theenumi.\arabic*}, align=left, leftmargin=*]
            \item The app validates the videos. Refers to \textit{US\_2}.
            \item The user is able to see the result of the test.  Refers to \textit{US\_2.4, US\_2.6, US\_2.7}.
        \end{enumerate}
    \item Migrating to React Native or PWA
    \item Using custom Deep Learning model or improving the existing one
\end{enumerate}
  
\section{Tool's selection}
The main reason why I chose these technologies was I already had some experience with them.
This way, I could minimize the risk of the development process and get the first MVP (Minimum Viable Product) as soon as possible.
In addition, all these technologies are standards in industry and learning more about them will allow me to fit better in the market.
The tools are listed on table \ref{table:analysis_tools_selection}. \\ 

\begin{table}[h]
    \centering
    \resizebox{\textwidth}{!}{
    \begin{tabular}{|p{3cm}|p{2cm}|p{4cm}|p{2cm}|p{4cm}|}
        \cline{1-5} Type & Tool & Description & Experience & Reason to choose it       \\
        \hline \multirow{5}{*}{Version control}   & Git     & Program to manage the different versions of a directory   & Y & Standard and experience \\
        \cline{2-5}                               & GitHub  & To host my git repository                                 & Y & Standard, experience and ability to use GitHub's projects and CI tools \\
        \hline \multirow{5}{*}{Backend}   & ASP.NET core   & C-Sharp framework to create REST API & Y & Standard and experience \\
        \cline{2-5}                       & Flask          & Python framework to create REST API  & N & Standard and ease to integrate with PyTorch \\
        \hline \multirow{5}{*}{Frontend}  & React.js    & Javascript library used to create SPA applications & Y & Standard and experience \\
        \cline{2-5}                       & Typescript  & Programming language                               & Y & Standard, experience, typed language \\
        \hline \multirow{5}{*}{Testing}  & xUnit    & Library to test ASP.NET applications & N & I had a project from Vilnius University where I had to use it and it was a recommendation of my teacher \\
        \cline{2-5}                      & Postman  & Application to do API calls, load tests, simple unit tests and integration tests to REST APIs & Y & There is a new feature that I want to try \\
        \cline{2-5}                      & Chai.js  & Library to do javascript tests. & N & Legacy by Postman \\
        \hline \multirow{5}{*}{Documentation}  & Draw.io    & Web application to design graphs and diagrams & Y & It does not need any license and it is intuitive \\
        \cline{2-5}                      & Swagger  & library to document the endpoints of the application & Y & I prefer it rather than using other's such as Postman due to its integration in the language \\
        \cline{2-5}                      & GitHub projects  & In order to do my planification, set up my sprints and follow the development process & Y & I'm already using GitHub \\
        \hline \multirow{5}{*}{Database}  & Microsoft SQL Server &  DBMS from Microsoft & Y & Experience and ease to integrate with ASP.NET Core \\
        \cline{2-5}                      & Entity framework  & ORM to work with databases for .NET applications & Y & Experience, standard, documentation and good integration \\
        \cline{2-5}                      & Azure data studio  & Program to connect to an existing DB, execute SQL statements, edit tuples... & Y & I prefer to work from a GUI \\
        \hline \multirow{5}{*}{Styling}  & Figma & Application to design mockups and high fidelity projects & Y & Standard, experience and easy to use \\
        \cline{2-5}                      & Material.ui & components library from Google & N & They take accessibility into account, it is stable and good looking \\
        \hline \multirow{5}{*}{CI}  & Microsoft SQL Server &  DBMS from Microsoft & Y & Experience and ease to integrate with ASP.NET Core \\
        \hline
    \end{tabular}
    }
\caption{Tool's selection analysis}
\label{table:analysis_tools_selection}
\end{table}
