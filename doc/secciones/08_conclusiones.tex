\chapter{Conclusions and future work}

This project has served me as my presentation letter to the job market and has given me the opportunity to learn industry-standard 
technologies and development methodologies used in companies nowadays. This is due to my attitude towards the project itself, as I 
did not want to simply meet the requirements of the project, but to do a job that can really serve people (a social purpose and a 
usable product) and that allows me to learn cutting-edge technologies and choose the branch of computer science that I find the most 
attractive (frontend, backend, mobile development or data science).
In addition to fulfilling the technical requirements to deliver this work, I have fulfilled my own personal and professional goals. \\

The first stages of the development process of this project were useful for me to learn more about deep learning, since in my study-branch 
(Software Development) we do not study this area of knowledge. Also, it helped me to learn how to read scientific articles and develop my 
project ideation, as the proposal for the development of the project was suggested by me.
The later stages helped me to learn which technologies were used in the industry and to focus on those that were really going to solve my 
problem without complicating my task. 

After a whole development process in which I have learned to use version control, divide a project into milestones and user stories, 
develop prototypes, create an API Rest, consume an API from a frontend, create a microservice that uses a machine learning model and 
perform unit and integration tests, I think I have grown as a developer. I have applied many concepts learned in my degree and I have 
learned to use them in a real environment. Also, personally, I have more confidence to be able to transform an idea into a software product, 
I feel better having developed a product that can help people at risk of exclusion and it has allowed me to find my first job. \\

Although the overall balance is positive, I am critical and there are some decisions I would have made. 
One of them is to avoid using beta versions of products. Although I used it to learn and because I found it attractive, it was an unnecessary 
cost for me to change every time the version was updated.
Another decision I would rethink would be to use a test-driven methodology, such as TDD (Test Driven Development). 
This is because in the end I implemented unit tests, but it would have saved me manual debugging time.
Also, I was lucky to have had time to dedicate to the work. I would have chosen a topic that covered fewer technologies and topics, 
because, if I had been in another situation, it would have been unmanageable to complete it. \\

From this work, the scientific community can gain a lot of value. 
The first contribution of this work for future works is that it can make use of the \textbf{literature review} \ref{LiteratureReview} that I carry out, where I analyze 
the state of art and provide insights about the main datasets, models and approaches that have already been performed. 
Secondly, the scientific community can draw a critic, and this is the \textbf{lack of public datasets} on this topic in other languages 
such as Spanish. 
Thirdly, from this work one of the largest public large-scale datasets will be tested \ref{WLASL}, so it can be a trial in a \textbf{real environment} 
to prove its efficiency and effectiveness.
Finally, \textbf{ideas} are offered for new applications to help the deaf-and-dumb community (integration of such a model at a conversational 
level in video call applications such as Jitsi \cite{Jitsi} so that non-signing users and signing users can interact in the same meeting).