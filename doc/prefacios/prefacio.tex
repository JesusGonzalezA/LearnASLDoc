\thispagestyle{empty}

\begin{center}
{\large\bfseries Learn ASL \\ Application to learn American Sign Language integrating a deep learning based model.}\\
\end{center}
\begin{center}
Jesús González Álvarez\\
\end{center}

%\vspace{0.7cm}

\vspace{0.5cm}
\noindent{\textbf{Keywords}: \textit{open source}}, \textit{deep learning}, \textit{sign language recognition}, \textit{integration of deep learning models in web applications}
\vspace{0.7cm}

\noindent{\textbf{Abstract}}\\

\bigskip
The main objective of the project is to develop a computer system to help people learn to sign. The system will be able to generate tests so that users can test their skills. In addition, it should store information regarding the results in order to be able to perform personalized tests and generate reports of their progress. It should also be able to recognize, through videos provided by the users, if the signed words correspond to those requested in the test, and thus be able to validate in a practical way if the user is executing them correctly. The application can be accessed as long as the user has a device with an Internet connection and a webcam.

\bigskip
Since sign language is not international, and taking into account the information available, the project will be limited to American Sign Language and a subset of English natural language.

\bigskip
The system will consist of a web module that will include user management, test generation, results management and generation of analytical information. This module will interact with another module based on artificial intelligence (deep learning) that will be able to process video fragments and automatically determine the signified word in the video. This second module will be used to process the sign language input from the users, and with the result produced by it, evaluate the correctness. This module will integrate an existing deep learning based model.
	

\cleardoublepage

\begin{center}
	{\large\bfseries Learn ASL \\ Aplicación para aprender Lenguaje de Signos Americano integrando un modelo de aprendizaje automático.}\\
\end{center}
\begin{center}
	Jesús González Álvarez\\
\end{center}
\vspace{0.5cm}
\noindent{\textbf{Palabras clave}: \textit{software libre}, \textit{aprendizaje profundo}, \textit{reconocimiento de lenguaje de signos}, \textit{integración de modelos de aprendizaje profundo en aplicaciones web} }
\vspace{0.7cm}

\noindent{\textbf{Resumen}}\\

\bigskip
El proyecto tiene como principal objetivo desarrollar un sistema informático para ayudar a las personas a aprender a signar. El sistema será capaz de generar tests para que los usuarios puedan poner a prueba sus habilidades. Además, deberá de guardar información relativa a los resultados para poder realizar tests personalizados y generar informes de sus avances. También deberá de ser capaz de reconocer, a través de vídeos proporcionados por los usuarios, si las palabras signadas se corresponden con las solicitadas en el test, y así poder validar de forma práctica si el usuario las está ejecutando correctamente. Se podrá acceder a la aplicación siempre que el usuario disponga de un dispositivo con conexión a Internet y una cámara web.

\bigskip
Dado que el lenguaje de signos no es internacional, y teniendo en cuenta la información disponible, el proyecto se limitará al lenguaje de signos americano (American Sign Language) y a un subconjunto del lenguaje natural en inglés.

\bigskip
El sistema estará compuesto por un módulo web que incluirá la gestión de usuarios, generación de tests, gestión de resultados y generación de información analítica. Este módulo deberá interactuar con otro basado en inteligencia artificial (aprendizaje profundo) que será capaz de procesar fragmentos de vídeo y determinar automáticamente la palabra signada en el mismo. Este segundo módulo será el utilizado para procesar las entradas en lenguaje de signos de los usuarios, y con el resultado producido por el mismo, evaluar la correctitud. Este módulo integrará un modelo basado en aprendizaje profundo ya existente.

\newpage
\thispagestyle{empty}
\
\vspace{3cm}

\noindent\rule[-1ex]{\textwidth}{2pt}\\[4.5ex]

Yo, \textbf{Jesús González Álvarez}, alumno de la titulación \textbf{Ingeniería Informática} de la \textbf{Universidad de Granada}, autorizo la ubicación de la siguiente copia de mi Trabajo Fin de Grado \textit{Learn ASL} en la biblioteca del centro para que pueda ser consultada por las personas que lo deseen.

\bigskip
El documento en formato {\tt LaTeX} se puede encontrar en el siguiente repositorio de {\tt GitHub}: \url{https://github.com/JesusGonzalezA/LearnASLDoc}.

\vspace{7.5cm}

\noindent Fdo: \textbf{Jesús González Álvarez}

\vspace{2cm}

\begin{flushright}
Granada, a 31 Junio de 2022
\end{flushright}

\newpage
\thispagestyle{empty}
\
\vspace{3cm}

\noindent\rule[-1ex]{\textwidth}{2pt}\\[4.5ex]

D. \textbf{Miguel Lastra Leidinger}, Profesor del Departamento Ingeniería del Software de la Universidad de Granada.


\vspace{0.5cm}

\textbf{Informo:}

\vspace{0.5cm}

Que el presente trabajo, titulado \textit{\textbf{Learn ASL}},
ha sido realizado bajo mi supervisión por \textbf{Jesús González Álvarez}, y autorizo la defensa de dicho trabajo ante el tribunal
que corresponda.

\vspace{0.5cm}

Y para que conste, expiden y firman el presente informe en Granada a Junio de 2022.

\vspace{1cm}

\textbf{El director: }

\vspace{4cm}

\noindent Fdo: \textbf{Miguel Lastra Leidinger}

\vspace{2cm}

\begin{flushright}
Granada, a 31 Junio de 2022
\end{flushright}

\chapter*{Acknowledgments}
\thispagestyle{empty}

\vspace{1cm}

\noindent To my family, who gave me the opportunity to study and values that I really appreciate.

\bigskip
\noindent To all my teachers, from collegue to the university, for their dedication and knowledge.

\bigskip
\noindent To my tutor, Miguel Lastra Leidinger, for believing in me, for his time and special dedication, for his curiosity and intention to help others.






