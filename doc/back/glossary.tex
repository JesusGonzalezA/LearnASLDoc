\chapter{Glossary}

\textbf{Backend}: part of a web application that the user does not see. It is responsible of replying to every user request and doing an action on every one. Among these actions it could store data, send stored data, communicate with other service...
\bigskip

\textbf{Frontend}: part of a web application that the user really see. It is the layer between the user and the backend.
\bigskip

\textbf{HTTP}: \textit{HyperText Transfer Protocol}, it enables the communication between client and server. 
\bigskip

\textbf{HTTPS}: \textit{HTTP Secure}. Encrypted version of HTTP.
\bigskip

\textbf{URL}: \textit{Uniform Resource Locator}, location of an object on a computer network.
\bigskip

\textbf{Header}: field of an HTTP request or response to send additional information.
\bigskip

\textbf{Body}: payload of an HTTP request or response.
\bigskip

\textbf{Metadata}: information about the thing that the client retrieves from an HTTP response.
\bigskip

\textbf{Form-data}: interface to construct a set of key/value pair to send information in an HTTP request.
\bigskip

\textbf{Params}: name-value pairs to request particular resources from a web server using HTTP.
\bigskip

\textbf{Map}: data structure that stores key-value pairs. There cannot be duplicate keys. The keys are used to access the element and its data (value).
\bigskip

\textbf{Lifetime}: also called \textit{life cycle}. Time between an object's creation and its destruction.
\bigskip

\textbf{Authorization}: process of verifying what specific resources a user has acess to.
\bigskip

\textbf{Authentication}: process of verifying who someone is.
\bigskip

\textbf{CORS}: \textit{Cross-Origin Resource Sharing}, is an HTTP-header based mechanism that allows a server to indicate any origin.
\bigskip

\textbf{State management}: in a Web UI Framework, it enables the application to remember a user interface state and apply it. Therefore, the UI of the application would be different according of the current state of the application.
\bigskip

\textbf{UI}: \textit{User Interface}, is the point of human-computer interaction and communication in a device.
\bigskip

\textbf{Component}: a part or element of a larger whole. In Web development, it refers to a part of the webpage. It can be reusable and it has its own responsability and state.
\bigskip

\textbf{Redux}: design pattern to manage the state of an application. 
\bigskip

\textbf{REST}: \textit{REpresentational State Transfer} is a software architectural style that was created to guide the design and development of the architecture for the WWW. It realies on a stateless, client-server protocol.
\bigskip

\textbf{API}: \textit{Application Programming Interface} is a software intermediary that allows two applications to communicate with each other.
\bigskip

\textbf{Design pattern}: in software design, it is a typical solution to resolve common problems. It is like a blueprint that you can customize to solve a particular problem in your application.
\bigskip

\textbf{Software Architecture}: organization of a system. It includes all components, how they interact with each other, the evolution...
\bigskip

\textbf{Software}: it is a set of instructions, data or programs used to operate computers.
\bigskip

\textbf{Clean Architecture}: is a software design philosophy that separates the elements of a design into ring levels.
\bigskip

\textbf{Scrum}: it is a framework used for project management that emphasizes teamwork, accountability and iterative progress toward a well-defined goal.
\bigskip

\textbf{Sprint}: it is a short, time-boxed period when a scrum team works to complete a set amount of work.
\bigskip

\textbf{State of art}: refers to the highest level of development that has been achieved to date in a design, process, material or technique and is a key point in any industrial engineering project.
\bigskip

\textbf{Product backlog}: it is a list of the new features, changes to existing features, bug fixes, infrastructure changes or other activities that a team may deliver in order to achieve a specific outcome. 
\bigskip

\textbf{Bug}: it is an error, flaw or fault in computer software that causes it to produce an incorrect or unexpected result, or to behave in unintended ways.
\bigskip

\textbf{Milestone}: it is a marker of a stage in a project. 
\bigskip

\textbf{MVP}: \textit{Minimum Viable Product}, is a product with enough features to attract early-adopter customers and validate the product idea. It is the first product to develop in order to get early feedback from real users.
\bigskip

\textbf{Kanban}: it is a method to manage work. A kanban table could divide the tasks in \textit{To do}, \textit{Doing}, \textit{To review}, \textit{Reviewing}, \textit{Done}. 
\bigskip

\textbf{User story}: it is a tool in agile development used to capture a description of a task from a user's perspective.
\bigskip

\textbf{Agile}: method of project management.
\bigskip

\textbf{Unit test}: software development process in which the smallest components and parts of the application (units) are tested.
\bigskip

\textbf{Integration test}: software development process in which the units are tested combined.
\bigskip

\textbf{Framework}: it is a standard, abstraction or template that gives to developers all the tool required to achieve a software product.
\bigskip

\textbf{Continuous integration}: software development practice in which developers merge their code changes into a repository, after automated builds and tests.
\bigskip

\textbf{Repository}: it is a central place to store, aggregate data and change data in a mainteined and organized way.
\bigskip

\textbf{Deep Learning}: type of machine learning and artificial intelligence that imitates the way humans gain certain types of knowledge. It includes statistics and predictive modelling.
\bigskip

\textbf{Model}: the deep learning element that performs classification or predictive tasks. 
\bigskip

\textbf{Dataset}: collection of data.
\bigskip

\textbf{Open source}: software that people can modify, see, share and use because its design and development is publicly accesible.
\bigskip

\textbf{Convolution}: mathematical operation on two functions that produces a third function that expresses how the shape of one is modified by the other.
\bigskip

\textbf{Layer}: it is a structure in a deep learning model that take information and pass information to the next layer.
\bigskip

\textbf{PWA}: \textit{Progressive Web Application}. It refers to a very specific type of web application, that can be downloaded using the browser and that can cache the information in order to be used offline, use operative system calls and can almost be seen as a native application, although it runs in the browser.
\bigskip